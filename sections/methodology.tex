% filepath: c:\Users\gabri\Documents\Ceub\tcc\sections\methodology.tex
\chapter{Metodologia}

\section{Visão Geral da Metodologia}
A metodologia deste trabalho está estruturada em duas frentes principais: a otimização de asas, considerando as fases de decolagem e cruzeiro, e a otimização de hélices, com parâmetros definidos pelo usuário. O fluxo metodológico foi delineado com base em referências clássicas e recentes sobre otimização aerodinâmica assistida por inteligência artificial \cite{oliveira2023, hasan2024, wu2024}.

\section{Automação da Obtenção de Perfis}
\label{sec:automacao}
A obtenção dos perfis aerodinâmicos é realizada de forma automatizada via \textit{web scraping} do repositório Airfoil Tools (\url{http://airfoiltools.com/}), conforme metodologia empregada em trabalhos recentes \cite{oliveira2023}. Após o download, é executada uma etapa de filtragem e validação dos dados, removendo duplicatas, perfis incompletos ou inconsistentes, e garantindo a integridade do conjunto de dados para análise posterior. Essa etapa é fundamental para evitar vieses e garantir a robustez dos resultados \cite{abbott1959theory}.

\section{Metodologia para Otimização de Asas}
\subsection{Definição dos Inputs}
\label{subsec:inputs}
Os principais parâmetros de entrada (inputs) para a otimização das asas incluem: velocidade de voo, altitude, peso da aeronave, corda média aerodinâmica, ângulos e proporções dos flaps para decolagem. Todos os dados são padronizados em unidades do Sistema Internacional (SI) e validados quanto à consistência, seguindo as melhores práticas de engenharia \cite{anderson2017fundamentals, raymer2018aircraft}.

\subsection{Pré-processamento e Seleção Inicial}
\label{sub:pontuacao}
Cada perfil é testado nas duas fases de voo mais críticas para o desempenho global: decolagem e cruzeiro. Os testes são realizados utilizando a ferramenta XFOIL, que permite a obtenção dos coeficientes aerodinâmicos em diferentes condições de contorno \cite{drela1989xfoil}. A seleção dos 1\% melhores perfis é baseada em uma pontuação composta:
\[
\text{Score} = \left| \frac{\text{S}_{\text{decolagem}}}{\text{S}_{\text{cruzeiro}}} - 1 \right|^{w_1} \cdot \left(\frac{C_L}{C_D}\right)_{\text{cruzeiro}}^{w_2}
\]
Onde \(w_1\) e \(w_2\) são pesos ajustáveis conforme a importância relativa de cada critério. Os pesos serão ajustados para que ambos os valores sejam considerados na otimização, evitando que o processo foque apenas em um dos objetivos. Recomenda-se a realização de análise de sensibilidade para definir os pesos ideais, conforme sugerido em \cite{oliveira2023, hasan2024}.
As áreas necessárias para decolagem e cruzeiro são calculadas no ponto em que a razão \(C_L/C_D\) é máxima para cada fase, utilizando os parâmetros de entrada definidos na Seção~\ref{subsec:inputs}. O cálculo da área é realizado a partir da Equação~\eqref{eq:sustentacao}, que relaciona a sustentação, a densidade do ar, a velocidade, a área da asa e o coeficiente de sustentação:

\[
 S = \frac{L}{\frac{1}{2} \rho V^2 C_L}
\]

Dessa forma, garante-se que a comparação entre os perfis seja feita considerando as condições mais favoráveis de desempenho aerodinâmico em cada fase do voo.

\subsection{Otimização com IA}
\label{subsec:otimizacao}
O número de pontos a serem otimizados em cada perfil é definido conforme a complexidade geométrica e o grau de liberdade desejado.

No contexto deste trabalho, a inteligência artificial (IA) é empregada de duas formas principais: redes neurais e algoritmos genéticos. As redes neurais são utilizadas para prever rapidamente os coeficientes aerodinâmicos (\(C_L\) e \(C_D\)) de novos perfis, após treinamento com dados gerados pelo XFOIL. Isso permite acelerar a avaliação de milhares de geometrias, reduzindo o custo computacional do processo de otimização \cite{wu2024, goodfellow2016deep}. Além disso, redes neurais podem ser integradas ao processo evolutivo, servindo como função de avaliação (fitness) ou como geradoras de novos candidatos a partir de parâmetros geométricos.

A escolha pelo uso de uma rede neural (RN) artificial, em vez de testar exaustivamente cada perfil aerodinâmico diretamente no XFOIL, justifica-se principalmente pelo ganho significativo em eficiência computacional e tempo de processamento. O XFOIL, embora seja uma ferramenta precisa e amplamente utilizada para análise de perfis aerodinâmicos, demanda tempo considerável para simular milhares de geometrias diferentes, especialmente em processos iterativos de otimização, como ocorre em algoritmos evolutivos.

Ao treinar uma RN com dados previamente gerados pelo XFOIL, é possível criar um modelo preditivo capaz de estimar rapidamente os coeficientes aerodinâmicos (\(C_L\), \(C_D\)) de novos perfis, reduzindo drasticamente o número de chamadas ao simulador. Isso permite explorar um espaço de soluções muito maior em menos tempo, viabilizando a otimização de perfis com múltiplos parâmetros e restrições. Além disso, a RN pode ser continuamente aprimorada com novos dados, aumentando sua precisão ao longo do processo.

Essa abordagem é respaldada na literatura, que destaca o uso de modelos de aprendizado de máquina para substituir simulações computacionalmente custosas em problemas de engenharia, acelerando o desenvolvimento e mantendo boa acurácia \cite{goodfellow2016deep, wu2024}. Portanto, a integração de redes neurais ao fluxo de trabalho representa uma solução eficiente, escalável e alinhada às tendências atuais em otimização aerodinâmica assistida por inteligência artificial.

Já os algoritmos genéticos (AGs) são aplicados para explorar o espaço de soluções e identificar perfis ótimos. Cada indivíduo da população representa um perfil completo, codificado por seus parâmetros geométricos. O processo evolutivo envolve:

\begin{itemize}
    \item \textbf{População inicial}: gerada aleatoriamente ou a partir de perfis conhecidos, garantindo diversidade, como explicado em \ref{sub:pontuacao}.
    \item \textbf{Seleção}: indivíduos com melhor desempenho (maior fitness) são escolhidos para reprodução. Métodos comuns incluem roleta, torneio e elitismo, sendo possível ajustar a pressão seletiva conforme desejado \cite{goldberg1989genetic}.
    \item \textbf{Cruzamento (crossover)}: combina características de dois ou mais indivíduos para gerar descendentes, promovendo a exploração do espaço de busca.
    \item \textbf{Mutação}: pequenas alterações aleatórias são aplicadas aos descendentes para evitar estagnação em ótimos locais. Os principais parâmetros de mutação são:
    \begin{itemize}
        \item \textbf{Chance de mutação}: probabilidade de um gene sofrer mutação, tipicamente entre 1\% e 10\%.
        \item \textbf{Porcentagem máxima de mutação}: define o limite da alteração em cada parâmetro, evitando mudanças drásticas que possam inviabilizar o perfil.
    \end{itemize}
    \item \textbf{Critério de parada}: o algoritmo é interrompido após um número fixo de gerações ou quando a melhoria do score se torna insignificante \cite{back1996evolutionary, oliveira2023}.
\end{itemize}

Testes preliminares são realizados para definir o número ideal de gerações, o tamanho da população, a taxa de cruzamento e de mutação, bem como para analisar a convergência dos algoritmos \cite{hasan2024, wu2024}. A combinação de IA e algoritmos evolutivos permite uma busca eficiente por soluções ótimas, mesmo em espaços de alta dimensionalidade e com múltiplos objetivos, como ocorre na otimização de perfis aerodinâmicos.

\subsection{Fluxo Metodológico da Otimização}
O processo de otimização segue os seguintes passos principais:

\begin{enumerate}
    \item \textbf{Treinamento da Rede Neural:} Simultaneamente à avaliação dos perfis para seleção dos 1\% melhores, todos os aerofólios obtidos (conforme descrito na Seção~\ref{sec:automacao}) são testados no XFOIL. Os dados de entrada e saída dessas simulações são utilizados para treinar a rede neural (RN), utilizando 80\% dos dados para treinamento e 20\% para validação. O treinamento é repetido caso o erro médio nos dados de teste ultrapasse 5\%.
    \item \textbf{Otimização Evolutiva:} Após o treinamento da rede neural, a otimização dos perfis de asa é conduzida por meio de um algoritmo evolutivo. Os critérios de seleção, cruzamento e mutação são aplicados para gerar novas soluções, e a avaliação dos indivíduos é realizada com base no Score, conforme definido na Seção~\ref{sub:pontuacao}. O processo evolutivo é repetido até que um dos critérios de parada seja alcançado como descrito na Seção~\ref{subsec:otimizacao}.
\end{enumerate}

Esse fluxo garante integração eficiente entre a avaliação baseada em simulação e a aceleração proporcionada pela inteligência artificial, otimizando o desempenho e a robustez dos resultados.

\section{Metodologia para Otimização de Hélices}

\subsection{Definição das Seções da Hélice}
A hélice é dividida em no mínimo duas seções: ponta e raiz, desconsiderando o perfil circular na raiz do hub  devido ao uso de materiais compósitos, o que permite maior flexibilidade estrutural e aerodinâmica \cite{raymer2018aircraft}. A possibilidade de expansão futura para três ou mais seções é prevista para aumentar a granularidade da otimização, conforme abordagens recentes \cite{wu2024}.

\subsection{Inputs do Usuário}
Os inputs necessários para a otimização  das hélices incluem: velocidade de voo, ângulo de incidência da hélice com a direção do voo, diâmetro, torque, RPM, altitude, número de pás e curva de tração (linear ou logaritimica). Recomenda-se o desenvolvimento de uma interface amigável para entrada desses dados, visando minimizar erros de entrada e facilitar a replicação dos experimentos \cite{wu2024, raymer2018aircraft}.

\subsection{Otimização das Seções}
Cada seção da hélice é otimizada individualmente, adotando ponta elíptica e proporcionalidade da área das seções à eficiência (\(C_L/C_D\)), conforme práticas consagradas em aerodinâmica \cite{anderson2017fundamentals, abbott1959theory}. A análise de sensibilidade para o número de seções e para a curva de tração é recomendada para avaliar o impacto dessas variáveis no desempenho final \cite{wu2024}. Primeiramente, será dividido o torque máximo ($\tau_{max}$) pelo número de pás que a hélice terá. Depois, será dividio o tamanho da pá em partes iguais, e cada aprte terá um perfil diferente. Com a distância para o Hub da hélice, RPM e a velocidade relativa, podemos calcular a velocidade de cada perfil. Com a velocidade de que cada perfil estará subimetida, podemos calcular o numero de Reynolds e consequentemente o arrasto máximo de cada seção, com a seguinte formula:
\[
    \tau_{\text{máx}} = \frac{D_{\text{máx}}\cdot r}{n_{\text{seções}}} 
\]
Sendo o $\tau$ o torque do motor, $D$ o arrasto, $r$ o raio para o hub e $n$ o número de seções.

\subsection{Considerações Computacionais}
A seleção prévia dos melhores perfis por seção pode reduzir significativamente o custo computacional, ao restringir o espaço de busca a soluções promissoras. No entanto, benchmarks devem ser realizados para comparar abordagens alternativas e garantir que não haja perda de diversidade ou qualidade nas soluções finais \cite{oliveira2023, hasan2024}.

\section{Validação e Análise dos Resultados}
A validação dos resultados é realizada por meio de comparação com métodos tradicionais de projeto, análise de desempenho aerodinâmico e testes de robustez dos modelos de IA. Recomenda-se a realização de testes de generalização, aplicando os modelos a cenários distintos para avaliar sua capacidade preditiva e adaptabilidade \cite{hasan2024, wu2024, goodfellow2016deep}.

\section{Possíveis Exclusões}
A exclusão do pouso é justificada pela predominância de dispositivos hipersustentadores e pelo menor impacto dessa fase no desempenho global da aeronave \cite{raymer2018aircraft, abbott1959theory}. Aspectos não abordados, como a raiz circular em hélices de materiais não compósitos, não serão detalhados, conforme delimitação do escopo deste trabalho.

\section{Sugestões Adicionais}
Recomenda-se a inclusão de fluxogramas e diagramas para ilustrar o fluxo metodológico, facilitando a compreensão e a replicação do processo. Além disso, a documentação detalhada do código e dos experimentos é fundamental para garantir a reprodutibilidade dos resultados, conforme boas práticas em pesquisas de otimização e IA \cite{goodfellow2016deep, oliveira2023}.
