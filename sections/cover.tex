\begin{titlepage}
\includegraphics[scale=0.15]{uniceub_logo}\\
\vspace{2cm}
    \begin{center}
        \vspace*{1cm}
        
        {\LARGE \textbf{Aplicação de Inteligência Artificial na Otimização de Geometrias Aerodinâmicas para Asas e Hélices de Aeronaves}}
        
        \vspace{0.5cm}
        TRABALHO DE CONCLUSÃO DE CURSO
        
        \vspace{1.5cm}
        por \\
        \vspace{1.5cm}
            Gabriel Holsback Dantas - \url{holsback@sempreceub.com} - RA: 22310899\\
        \vspace{1.0cm}
    \end{center}

    \vspace{1cm}
\noindent{
{
    {\bf Resumo:} 
    Este trabalho apresenta o desenvolvimento de uma Inteligência Artificial para otimização de perfis aerodinâmicos de asas e hélices, utilizando os softwares Xfoil e XFLR5. A metodologia empregada combina técnicas de aprendizado de máquina e otimização numérica para aprimorar a eficiência aerodinâmica, considerando diferentes fases do voo. Os resultados obtidos buscam contribuir para o avanço na análise e design de estruturas aeronáuticas, possibilitando melhorias significativas em desempenho e redução de arrasto.
    }
}
\vfill
\textcolor[rgb]{0.5,0.5,0.5}{
    \begin{flushleft}
    { \small
    Orientador: Nome Orientador\\
    Turma: UN2023/01\\
    Curso: Engenharia da Computação\\
    Campus: Asa Norte\\
    Turno: Noturno
    }
    \end{flushleft}
}
      
\end{titlepage}