\chapter{Introdução}

A evolução da mobilidade aérea tem enfrentado desafios significativos, impulsionados pela crescente demanda por eficiência energética, sustentabilidade e inovação tecnológica. No setor aeronáutico, a busca por aeronaves mais leves, eficientes e com melhor desempenho aerodinâmico é essencial para atender às exigências de um mercado em constante transformação. Seja em aplicações comerciais, drones, veículos aéreos não tripulados (VANTs) ou aeronaves de decolagem e pouso vertical (VTOL), a otimização de geometrias aerodinâmicas, como asas e hélices, desempenha um papel crucial na melhoria da eficiência global do sistema.

Tradicionalmente, o desenvolvimento de perfis aerodinâmicos envolve processos complexos e demorados, que dependem de simulações computacionais, testes em túnel de vento e iterações manuais realizadas por engenheiros especializados. Esses métodos, embora eficazes, apresentam limitações em termos de custo e tempo, especialmente para empresas de menor porte que buscam inovar no setor aeroespacial.

Nesse contexto, a inteligência artificial (IA) surge como uma solução disruptiva, oferecendo ferramentas avançadas para a otimização aerodinâmica. Técnicas como algoritmos genéticos e redes neurais têm se mostrado promissoras na exploração de geometrias otimizadas, permitindo a análise de milhares de configurações em um curto período de tempo. Essas abordagens possibilitam a identificação de soluções que maximizam a sustentação, minimizam o arrasto e otimizam o desempenho em diferentes fases do voo, como decolagem, cruzeiro e pouso.

Este trabalho tem como objetivo principal investigar a aplicação de inteligência artificial na otimização de perfis aerodinâmicos de asas e hélices, utilizando algoritmos genéticos e redes neurais integrados ao software Xfoil. A proposta busca automatizar o processo de análise e design aerodinâmico, explorando métricas como coeficientes de sustentação e arrasto, eficiência energética e estabilidade. Com isso, espera-se contribuir para a democratização de tecnologias avançadas no setor aeroespacial, promovendo inovações que possam beneficiar tanto grandes players quanto startups e pesquisadores independentes.

\section{Considerações Preliminares}

O desenvolvimento de perfis aerodinâmicos otimizados é um desafio multidisciplinar que envolve conhecimentos em aerodinâmica, engenharia computacional e inteligência artificial. A utilização de ferramentas como o Xfoil, amplamente reconhecido por sua capacidade de simular e analisar o desempenho de perfis aerodinâmicos, permite a integração de técnicas avançadas de otimização. Neste trabalho, algoritmos genéticos e redes neurais são explorados como métodos complementares para automatizar e aprimorar o processo de design aerodinâmico. Essas técnicas são particularmente relevantes devido à sua capacidade de lidar com problemas complexos e não lineares, características inerentes ao design aerodinâmico. 

Para o desenvolvimento das soluções de inteligência artificial, será utilizada a linguagem Python, com foco na implementação de algoritmos próprios de IA. O objetivo é minimizar o uso de bibliotecas específicas de IA, como TensorFlow ou Keras, priorizando o uso de bibliotecas matemáticas e de manipulação de dados, como NumPy e Pandas. Essa abordagem busca proporcionar maior controle sobre os algoritmos desenvolvidos, permitindo uma compreensão mais profunda dos métodos aplicados.

\section{Objetivos}

O objetivo principal deste trabalho é desenvolver um sistema automatizado para a otimização de perfis aerodinâmicos utilizando inteligência artificial. Para isso, os seguintes objetivos específicos foram definidos:

\begin{itemize}
    \item Implementar algoritmos genéticos para explorar e identificar configurações aerodinâmicas otimizadas;
    \item Desenvolver redes neurais personalizadas para prever o desempenho aerodinâmico de perfis com base em dados gerados pelo Xfoil;
    \item Integrar as técnicas de IA ao fluxo de trabalho do Xfoil, permitindo uma análise eficiente e automatizada;
    \item Avaliar o desempenho das soluções otimizadas em termos de coeficientes de sustentação, arrasto e eficiência aerodinâmica;
    \item Comparar os resultados obtidos com métodos tradicionais de design aerodinâmico.
\end{itemize}

\section{Contribuições}

Este trabalho busca contribuir para o avanço do uso de inteligência artificial no setor aeroespacial, com foco nas seguintes áreas:

\begin{description}
    \item[Automatização do Design Aerodinâmico:] Reduzir o tempo e o esforço necessários para projetar perfis aerodinâmicos otimizados, democratizando o acesso a tecnologias avançadas.
    \item[Algoritmos Próprios:] Implementar soluções de IA personalizadas, minimizando a dependência de bibliotecas externas e promovendo maior controle sobre os métodos utilizados.
    \item[Integração de IA e Simulação:] Demonstrar a viabilidade de integrar algoritmos genéticos e redes neurais ao Xfoil, criando um fluxo de trabalho eficiente e acessível.
    \item[Inovação no Setor Aeroespacial:] Fornecer uma abordagem inovadora que possa ser aplicada tanto por grandes empresas quanto por startups e pesquisadores independentes.
    \item[Base para Pesquisas Futuras:] Estabelecer uma base metodológica que possa ser expandida para outras aplicações, como otimização de asas completas, hélices e turbinas eólicas.
\end{description}

Com essas contribuições, espera-se que este trabalho não apenas avance o estado da arte em otimização aerodinâmica, mas também inspire novas aplicações de inteligência artificial em engenharia aeroespacial.